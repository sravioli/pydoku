\documentclass[a4paper, 10pt]{article}

\usepackage[utf8]{inputenc}
\usepackage[italian]{babel}
\usepackage[T1]{fontenc}
\usepackage{csquotes}

\usepackage[colorlinks=true]{hyperref}

\usepackage{datetime2}
\usepackage{enumitem}

\usepackage{graphicx}
\usepackage{subcaption}


% change datetime2 settings
\DTMsetdatestyle{ddmmyyyy}
\DTMsetup{datesep=/}

\usepackage{minted}
\usemintedstyle{one-dark}

\usepackage{biblatex}
\addbibresource{citations.bib}



% count within section
\counterwithin{figure}{section}
\counterwithin{listing}{section}

\newcommand{\lenet}{LeNet\(5\)+}


\begin{document}

\title{\textsc{Risolutore di sudoku in Python con OpenCV e Keras}}
\author{Simone Fidanza}
\date{\today}


\maketitle

\begin{abstract}
    Per estrarre la griglia di un sudoku da un'immagine dello stesso è stata
    utilizzata la libreria OpenCV, in seguito le singole cifre sono state
    analizzate dalla Convolutional Neural Network ispirata al modello LeNet5.
    Dopo aver predetto le cifre e aver costruito una griglia artificiale di
    numeri, il sudoku viene risolto da un'algoritmo.
\end{abstract}


\section{Introduzione}\label{sec:introduzione}

L'obiettivo ultimo dell'applicazione è quello di risolvere un sudoku da
un'immagine. Per poter fare ciò sono necessari diversi passaggi:

\begin{enumerate}
    \item Fornire un'immagine in input contenente un sudoku;
    \item Localizzare la griglia del sudoku ed estrarla;
    \item Data la griglia, individuare ogni singola cella della stessa;
    \item Determinare se la cella contiene una cifra e se sì, effettuare il
        riconoscimento ottico dei caratteri (per brevità OCR);
    \item Risolvere il sudoku mediante un algoritmo;
    \item Mostrare il sudoku risolto all'utente.
\end{enumerate}

L'applicazione fa ampio utilizzo delle librerie OpenCV\footnote{acronimo in
lingua inglese di \emph{Open Source Computer Vision Library}
}~\cite{opencv_library}, TensorFlow~\cite{tensorflow2015-whitepaper},
Keras~\cite{chollet2015keras} e NumPy~\cite{harris2020array}.


\section{Localizzazione della griglia}

Come prima cosa, dopo aver ricevuto l'immagine di un sudoku in input, è
necessario individuare la griglia del sudoku e per farlo è stata utilizzata la
libreria OpenCV. L'immagine iniziale subisce una serie di passaggi, ovvero:

\begin{itemize}
    \item viene convertita in scala di grigi con
        \mintinline{py3}{cv2.cvtColor()};
    \item viene applicata una sfocatura Gaussiana con
        \mintinline{py3}{cv2.GaussianBlur()} per renderla più uniforme;
    \item viene segmentata con
        \mintinline{py3}{cv2.adaptiveThreshold()}\footnote{è stata preferita
        una soglia adattiva per conservare quanti dettagli possibile
        dell'immagine} per trasformarla in un'immagine binaria;
    \item viene invertita con \mintinline{py3}{cv2.bitwise_not()};
    \item viene infine erosa con \mintinline{py3}{cv2.erode()} e
        successivamente dilatata con \\ \mintinline{py3}{cv2.dilate()} per
        rimuovere il rumore presente nell'immagine.
\end{itemize}

\begin{figure}[ht]
    \def\subwidth{0.40}
    \def\imgwidth{0.70}
    \centering
    \begin{subfigure}[b]{\subwidth\linewidth}
        \centering
        \includegraphics[width=\imgwidth\linewidth]{imgs/input.png}
        \caption{Originale}\label{subfig:input}
        \vspace{4ex}
    \end{subfigure}%%
    \begin{subfigure}[b]{\subwidth\linewidth}
        \centering
        \includegraphics[width=\imgwidth\linewidth]{imgs/process_out.png}
        \caption{Pre-elaborazione}\label{subfig:preprocess}
        \vspace{4ex}
    \end{subfigure}
    \begin{subfigure}[b]{\subwidth\linewidth}
        \centering
        \includegraphics[width=\imgwidth\linewidth]{imgs/grid.png}
        \caption{Griglia}\label{subfig:grid}
    \end{subfigure}%%
    \begin{subfigure}[b]{\subwidth\linewidth}
        \centering
        \includegraphics[width=\imgwidth\linewidth]{imgs/warp.png}
        \caption{Deformazione}\label{subfig:warp}
    \end{subfigure}
    \caption{Processo di estrazione della griglia di un sudoku}\label{fig:extraction}
\end{figure}

Il risultato di questi passaggi è illustrato in figura~\ref{subfig:preprocess}.
Dopo che l'immagine è stata pre-elaborata, si può procedere all'estrazione
della griglia.

Innanzitutto è necessario individuare il bordo della griglia. È stato supposto
che questo sia il più grande tra i bordi che la funzione
\mintinline{py3}{cv2.findContours()} restituirà, come illustrato in
figura~\ref{subfig:grid}.
Approssimando il bordo con \\ \mintinline{py3}{cv2.approxPolyDP()} è possibile
ottenere gli angoli della griglia.
Usando sia quest'ultimi che \mintinline{py3}{cv2.getPerspectiveTransform()} che \mintinline{py3}{cv2.WarpPerspective()} l'immagine viene deformata e resa piana,
come illustrato in figura~\ref{subfig:warp}.


\section{Analisi delle celle}

Per determinare se una cella contiene un numero oppure non lo contiene è necessario innanzitutto
estrarre le singole celle dalla griglia piana. Per fare ciò, sono state divise
per \(9\) le dimensioni della griglia e sono state estratte \(81\) celle di tali
dimensioni. Attraverso un semplice algoritmo viene verificato se la cella
contiene un numero oppure no. Se la cella contiene un numero, questo viene
estratto e analizzato dalla rete neurale e la predizione viene inserita nella
lista rappresentante la griglia del sudoku. Se la cella non contiene un numero,
nella lista viene inserito il numero \(0\).


\section{Modello}

Per poter effettuare l'OCR è necessario un modello di Machine Learning. Il
modello proposto è un miglioramento della rete neurale
LeNet5~\cite{lecun1998gradient}. Il modello non è altro che una semplice
Convolutional Neural Network (CNN per brevità) che è stata allenata sul dataset
MNIST~\cite{deng2012mnist} che contiene \(60,000\) immagini per l'allenamento e
\(10,000\) per l'analisi. Il modello è illustrato in
figura~\ref{fig:architecture}. \\

\begin{figure}[h]
    \includegraphics[width=0.9\textwidth, trim = 1cm 4.5cm 0cm 1cm]{architecture.pdf}
    \caption{Architettura della rete neurale \lenet. Ogni piano rappresenta un
    layer della Rete Neurale.}\label{fig:architecture}
\end{figure}

Per allenare \lenet{} è stato utilizzato un tasso di apprendimento variabile.
Nel momento in cui il modello registra che l'apprendimento è ``stagnante'', il
tasso dello stesso viene ridotto di \(0.2\) attraverso la classe
\mintinline{py3}{ReduceLROnPlateau()} del modulo \mintinline{py3}{callbacks}
di \mintinline{py3}{keras}.
Utilizzando l'accorgimento precedente e allenando il modello per \(30\) epoche,
con una dimensione del lotto di \(86\), esso ha raggiunto un'accuratezza del
\(\approx 99.67\%\).

\section{Risoluzione del sudoku}

Dopo aver estratto le cifre dalla griglia e aver formato una griglia
``digitale'' del sudoku, quest'ultima viene passata alla funzione
\mintinline{py3}{SudokuSolver.solve()} che risolve il sudoku. Infine viene
stampata a schermo la soluzione.

\section{Limitazioni conosciute}

Poiché l'algoritmo di estrazione della griglia non è perfetto, qui di seguito le
limitazioni conosciute:

\begin{itemize}
    \item nel deformare e appiattire un'immagine già piana, questa viene ruotata di \(90^\circ\) in senso orario. Per ovviare a questo inconveniente, è stata aggiunta una rotazione in senso antiorario;
    \item nel determinare se una cella è priva o meno di numero, poiché l'algoritmo si basa sul numero di pixel bianchi, a volte l'algoritmo può ignorare celle che contengono dei numeri. Per questo motivo la funzione \mintinline{py3}{SudokuExtractor.construct_board()} accetta un parametro opzionale che è proprio il valore minimo di pixel bianchi;
    \item nel predirre le cifre, la rete neurale confonde il numero \(1\) col numero \(7\) e raramente il numero \(6\) col numero \(5\) o \(8\)\footnote{questo accade quando la parte superiore del numero \(6\) è troppo incurvata verso il basso}
\end{itemize}


\bigskip
\printbibliography

\end{document}
